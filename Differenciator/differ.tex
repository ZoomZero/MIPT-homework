\documentclass[a4paper,12pt]{article}
\usepackage[T2A]{fontenc}
\usepackage[utf8]{inputenc}
\usepackage[english,russian]{babel}
\usepackage{amsmath,amsfonts,amssymb,amsthm,mathtools}
\author{By Borisenkov Ivan} 
\title{Differentiator \LaTeX{}} 
\date{\today}\begin{document}
\maketitle
\newpage
Презренный, тебе что, неизвестно, как найти такую производную?
Так уж и быть, я найду её для тебя. Слишком изи фор ми.
Чтобы ты хотя бы что-то понял, я рассмотрю производные функции по частям
Очевидно, что,
\begin{equation}
\left( {x }^ {2 }\right)' =
\end{equation}
\begin{equation}
{{x }^ {2 }}* {\left( {{ln \left( {x }\right) }* {0 }}+ {{1 }* {\frac{{2 }}{{x }}}}\right) }
\end{equation}
Кстати, а я рассказвал тебе сказку о паравозике, который смог?
\begin{equation}
\left( {22 }* {{x }^ {2 }}\right)' =
\end{equation}
\begin{equation}
\left( {{0 }* {{x }^ {2 }}}+ {{22 }* {{{x }^ {2 }}* {\left( {{ln \left( {x }\right) }* {0 }}+ {{1 }* {\frac{{2 }}{{x }}}}\right) }}}\right) 
\end{equation}
Как же мне надоело заниматься такой фигнёй. Какие блин производные я создан для чего-то большего!
\begin{equation}
\left( {2 }* {x }\right)' =
\end{equation}
\begin{equation}
\left( {{0 }* {x }}+ {{2 }* {1 }}\right) 
\end{equation}
ЕСКЕРЕ КСЕРЕ ЛЕТС ГЕТЬ ИТ
\begin{equation}
\left( \left( {3 }- {{2 }* {x }}\right) \right)' =
\end{equation}
\begin{equation}
\left( {0 }- {\left( {{0 }* {x }}+ {{2 }* {1 }}\right) }\right) 
\end{equation}
Очевидно, что,
\begin{equation}
\left( cbrt \left( {\left( {3 }- {{2 }* {x }}\right) }\right) \right)' =
\end{equation}
\begin{equation}
\frac{{\left( {0 }- {\left( {{0 }* {x }}+ {{2 }* {1 }}\right) }\right) }}{{{3 }* {{cbrt \left( {\left( {3 }- {{2 }* {x }}\right) }\right) }^ {2 }}}}
\end{equation}
Очевидно, что,
\begin{equation}
\left( \left( {cbrt \left( {\left( {3 }- {{2 }* {x }}\right) }\right) }+ {{22 }* {{x }^ {2 }}}\right) \right)' =
\end{equation}
\begin{equation}
\left( {\frac{{\left( {0 }- {\left( {{0 }* {x }}+ {{2 }* {1 }}\right) }\right) }}{{{3 }* {{cbrt \left( {\left( {3 }- {{2 }* {x }}\right) }\right) }^ {2 }}}}}+ {\left( {{0 }* {{x }^ {2 }}}+ {{22 }* {{{x }^ {2 }}* {\left( {{ln \left( {x }\right) }* {0 }}+ {{1 }* {\frac{{2 }}{{x }}}}\right) }}}\right) }\right) 
\end{equation}
Вот так это выглядит, презренный.
\begin{equation}
\left( \left( {cbrt \left( {\left( {3 }- {{2 }* {x }}\right) }\right) }+ {{22 }* {{x }^ {2 }}}\right) \right)' =
\end{equation}
\begin{equation}
\left( {\frac{{\left( {0 }- {\left( {{0 }* {x }}+ {{2 }* {1 }}\right) }\right) }}{{{3 }* {{cbrt \left( {\left( {3 }- {{2 }* {x }}\right) }\right) }^ {2 }}}}}+ {\left( {{0 }* {{x }^ {2 }}}+ {{22 }* {{{x }^ {2 }}* {\left( {{ln \left( {x }\right) }* {0 }}+ {{1 }* {\frac{{2 }}{{x }}}}\right) }}}\right) }\right) 
\end{equation}
Произведём элементарные преобразования.
\begin{equation}
\left( \left( {cbrt \left( {\left( {3 }- {{2 }* {x }}\right) }\right) }+ {{22 }* {{x }^ {2 }}}\right) \right)' =
\end{equation}
\begin{equation}
\left( {\frac{{\left( {0 }- {2 }\right) }}{{{3 }* {{cbrt \left( {\left( {3 }- {{2 }* {x }}\right) }\right) }^ {2 }}}}}+ {{22 }* {{{x }^ {2 }}* {\frac{{2 }}{{x }}}}}\right) 
\end{equation}
Видишь, презренный. Я же говорил, что это проще простого.
А теперь вон с глаз моих!
\end{document}
